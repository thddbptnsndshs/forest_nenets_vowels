% !TEX TS-program = XeLaTeX
\documentclass[10 pt, handout]{beamer}

% BEAMER SLIDES SETUP

\usecolortheme{rose}
\usefonttheme{professionalfonts}
\usefonttheme{serif}
\definecolor{antiquefuchsia}{rgb}{0.57, 0.36, 0.51}
\definecolor{bazaar}{rgb}{0.6, 0.47, 0.48}
\definecolor{cadet}{rgb}{0.33, 0.41, 0.47}
\setbeamercolor{titlelike}{fg=antiquefuchsia}
\setbeamercolor{structure}{fg=bazaar}
\usepackage{hyperref}
\hypersetup{
    colorlinks=true,
    linkcolor=cadet,
    citecolor=cadet,
    filecolor=cadet,
    urlcolor=cadet,
}
\usepackage{tipa}

% STYLE

\usepackage{float}
\usepackage{pifont}
\usepackage{graphicx}
\graphicspath{ {./images/} }
\usepackage{subfig}
\usepackage{enumerate}
\usepackage[normalem]{ulem} % underlining
\usepackage{multirow}  
\usepackage{booktabs} % tables
%\PassOptionsToPackage{table}{xcolor}% coloring tables
\usepackage{colortbl}

\AtBeginSection[]{ % section header slides
  \begin{frame}
  \vfill
  \centering
  \begin{beamercolorbox}[sep=8pt,center,shadow=true,rounded=true]{title}
    \usebeamerfont{title}\insertsectionhead\par%
  \end{beamercolorbox}
  \vfill
  \end{frame}
}

\setbeamertemplate{footline}[frame number] % slide numbers

% LANGUAGE + FONT
		    
\usepackage[english]{babel}
\usepackage[backend=biber,
                     style=unified]{biblatex}
\newcommand{\citeay}[2][]{
   \citeauthor{#2} (\citeyear[#1]{#2})}
\addbibresource{ref.bib}
\usepackage{fontspec}  
\setmainfont{Fira Sans}
%\setmainfont{TeX Gyre Termes}

% DRAWING

\usepackage{tikz}
\usepackage{tikz-qtree}
\usetikzlibrary{shapes.geometric}
\usetikzlibrary{trees,arrows}
\usetikzlibrary{positioning}
\usetikzlibrary{matrix}
\usetikzlibrary{tikzmark}
\usetikzlibrary{decorations.shapes}
\usetikzlibrary{shapes.misc}

% LINGUISTICS 

\usepackage{expex}
\usepackage[glossaries]{leipzig}
\makeglossaries
\newleipzig {evid} {evid} {evidentiality}
\newleipzig {cn} {cn} {connegative}
\newleipzig {sub} {sub} {subjective}
\newleipzig {gfs} {gfs} {general finite stem}
\newleipzig {hab} {hab} {habitual}
\newleipzig {prt} {prt} {preterite}

\title{Нейтрализация долгот \\в лесном ненецком языке} %\\ {\normalsize Forest Nenets vowel alternations}}
\author{Александра Шикунова (НИУ ВШЭ, Москва)}
\institute{XXI конференция по типологии и грамматике для молодых исследователей, Санкт-Петербург}
\date{21-11-2024}

\begin{document}

\begin{frame}
\titlepage
\end{frame}

\begin{frame}{Введение: нейтрализация}

	Безударные гласные часто теряют часть контрастных признаков
	\vspace*{1em}
	
	\begin{enumerate}[\ding{166}]
		\item Легко найти примеры нейтрализации по сегментным признакам 
		\item В русском языке пространство из 5 контрастных фонем в безударных слогах сужается до 3 или 2
		\item Может ли нейтрализация затрагивать супрасегментные признаки, такие, как долгота гласных?
	\end{enumerate}
	
\end{frame}

\begin{frame}{Лесной ненецкий язык}

	\begin{enumerate}[\ding{69}]
		\item Лесной ненецкий < самодийские < уральские
		\item Данные собраны методом элицитации в г. Тарко-Сале в 2024 году
	\end{enumerate}
	
	\begin{figure}[H]
		\includegraphics[width=.65\textwidth]{yanao-map}
		\hfill
		\includegraphics[width=.29\textwidth]{salminen-map}
		\label{}
		\caption*{\footnotesize Карта справа из \textcite{salminen2019}}
	\end{figure}
		
\end{frame}


\begin{frame}{Инвентарь гласных}

	\begin{enumerate}[\ding{102}]
		\item Гласные в ударном слоге противопоставлены по долготе
		\item В безударном слоге контраст нейтрализуется
		\item Также нейтрализуется контраст между гласными среднего и высокого подъема
	\end{enumerate}
	
	\vfill
	
\begin{minipage}{.45\textwidth}
\centering
Ударные слоги

\begin{table}[H]
\begin{tabular}{ccc}
ĭ i  &  &  ŭ u   \\
ĕ e  &    & ŏ o \\
æ̆ æ & ă a &    
\end{tabular}
\end{table}
\end{minipage}
%\hfill
\begin{minipage}{.5\textwidth}
\centering
Безударные слоги

\begin{table}[H]
\begin{tabular}{ccc}
\multirow{2}{*}{°} & i & u \\
                   & æ & a
\end{tabular}
\end{table}
\end{minipage}


\end{frame}

\begin{frame}{Ударение}

	\begin{enumerate}[\ding{87}]
		\item Ударение падает на нечетные слоги, но последний --- безударный
		
	\ex	ˈka.ta \hfill `бабушка'\\
		ˈta.pa.ta \hfill `указать'\\
		ˈta.pa.ˈta.ŋa \hfill `указать.{\Gfs}'	
	\xe
		
		\item Если ударение падает на открытый слог с краткой гласной, следующая согласная удлиняется (компенсаторная геминация)		

	\ex	ˈwă.ta [wătta] \hfill `крючок'\\
		ˈxă.ma [xămma] \hfill `шапка'\\
		ˈdʹĭλ´i [dʹĭλλi] \hfill `луна'
	\xe
	\end{enumerate}

\end{frame}

\begin{frame}{Шва} 

	Шва обозначается как /°/ и может иметь несколько поверхностных реализаций:
	\vspace*{1em}

	\begin{enumerate}[\ding{246}]
		\item Ноль либо [ĭ]
		
	\ex	tăλ.kăt° [tăλkăt $\sim$ tăλkătĭ] \hfill `мех.{\Abl}'\\
		măń° [măń $\sim$ măńĭ]  \hfill `я'
	\xe
		
		\item {[ă]} при прояснении в контексте C°C°
	\ex	λat° $+$ š° $\rightarrow$ λatăš° \hfill `ударить.{\Cvb}'\\
		kăλ° $+$ n° $\rightarrow$ kăλăn° \hfill `нож.{\Dat}'
	\xe
		
	\end{enumerate}

\end{frame}

\begin{frame}{Шва} 

	Шва обозначается как /°/ и может иметь несколько поверхностных реализаций:
	\vspace*{1em}

	\begin{enumerate}[\ding{117}]
		\item Удлинение и восстановление контраста между высоким и средним подъёмом в предыдущем слоге	

	\ex	ka.ta [kată] --- ka.ta.λ° [kataλ] \hfill `бабушка' --- `бабушка.{\Poss}.{\Ssg}'\\
		ńiʰ.ku [ńiʰkŭ] --- ˈńiʰ.ko.n° [ńiʰkon] \hfill `окунь' --- `окунь.{\Dat}' 
	\xe
		
		\item Геминация предыдущей согласной
			
	\ex dʹăλ.na.s° [dʹăλnas] \hfill `платье'\\
		kă.λ°.xă.na [kăλλxăn(n)a] \hfill `нож.{\Loc}'
	\xe
	\end{enumerate}

\end{frame}

\begin{frame}{Чередования гласных}

	\begin{enumerate}[\ding{96}]
		\item В безударных слогах гласные /е о/ переходят в /i u/
		\item Нейтрализуется контраст между высоким и средним подъёмом гласных
	\end{enumerate}

%	TODO: пример на e/i
%	\pex
%		\emph{wińi} --- \emph{wińi-t°} \hfill `{\Emph}.{\Neg} --- {\Emph}.{\Neg}-{\Fsg}'\\
%		\emph{} --- \emph{} \hfill ` --- '\\
%	\xe
		
%	\ex	\emph{dʹŭλ°nu} --- \emph{dʹŭλ°nuj°} \hfill `утро --- утренний'\\ % без швы можно?
%		\emph{ŋotaʰku} --- \emph{ŋotaʰkoj°} \hfill `дорожка --- дорожка.{\Poss}.{\Fsg}'
%	\xe
	\ex	\emph{xaλʹ°ʔńu} --- \emph{xaλʹ°ʔńuxŭt°} \hfill `яйцо --- яйцо.{\Abl}'\\ % можно
		\emph{xalaʰku} --- \emph{xalaʰkoxŏt°} \hfill `зверь --- зверь.{\Abl}'
	\xe

\end{frame}

\begin{frame}{Вопросы для инструментального исследования}

	Для лесного ненецкого языка (ЛН) практически не имеется акустических данных, которые помогли бы ответить на вопросы:
	\vspace*{1em}
	
	\begin{enumerate}[\ding{50}]
		\item Как длительности гласных на поверхности связаны с глубинными долготами?
		\item Чем определяется длительность гласных в безударных слогах?\\
			т.\ е.\ что такое ``нейтральная долгота''?
	\end{enumerate}

\end{frame}

			\section{Акустические данные} 

\begin{frame}{Данные}

	Полевая работа в г. Тарко-Сале (ЯНАО)

	\begin{enumerate}[\ding{68}]
		\item Июль 2024 (с опорой на данные, собранные в июне--июле 2023)
		\item 11 консультантов (3 мужчин, 8 женщин)
		\item Zoom H1n 48k 16bit
		\item Ручная аннотация в Praat \parencite{praat}
		\item Всего собрано 3906 слов
	\end{enumerate}

\end{frame}

\begin{frame}{Анкета}

	В списке слов для элицитации для гласных контролировались следующие переменные:

	\begin{enumerate}[\ding{77}]
		\item Структура слога: CV, CVC, CVV, CVVC
		\item Ударение: ударная/безударная
		\item Количество слогов в слове: односложное/многосложное
		\item Позиция слога: начальный, внутренний, конечный
		\item Подъём гласной:	
			\begin{enumerate}[\ding{163}]
				\item низкий /a, ă/
				\item средний /e, o, ĕ, ŏ/
				\item высокий /i, u, ĭ, ŭ/
			\end{enumerate}
	\end{enumerate}

\end{frame}

\begin{frame}{Долготы и длительности под ударением}

	В ударных слогах, как и ожидалось, слоги выстраиваются по длительности в порядке CVV > CVVC > CV > CVC % sync graph with hierarchy?
		
	\begin{figure}[H]
		\includegraphics[height=.7\paperheight]{polysyll\_dist}
	\end{figure}

\end{frame}

\begin{frame}{Безударные слоги}

	Безударные гласные (отмечены синим на графике)
		
	\begin{enumerate}[\ding{226}]
		\item Долгие: длиннее нейтральных
		\item Краткие: немного короче нейтральных
	\end{enumerate}
		
	\begin{figure}[H]
		\includegraphics[height=.45\paperheight]{unstressed\_polysyll\_dist}
		\hfill
		\includegraphics[height=.45\paperheight]{unstressed\_polysyll\_dist\_short}
	\end{figure}

\end{frame}

\begin{frame}{Вариация длительностей в безударных слогах}

	Безударные гласные сильно варьируются по длительности
	\vspace*{1em}

	\begin{enumerate}[\ding{102}]
		\item Между закрытыми и открытыми слогами: наличие коды сокращает гласную
	\end{enumerate}
	
	\begin{figure}[H]
		\includegraphics[height=.5\paperheight]{unstressed\_syll\_structure}
	\end{figure}

\end{frame}

\begin{frame}{Вариация длительностей в безударных слогах}

	Безударные гласные сильно варьируются по длительности
	\vspace*{1em}

	\begin{enumerate}[\ding{94}]
		\item В зависимости от распределения слогового веса в слове
		\item Длительность гласной во втором финальном закрытом слоге сильно увеличивается, если первый слог --- CV
	\end{enumerate}
	
\begin{table}[H]
\centering
\begin{tabular}{llrrr}
\toprule
 {word} & {segment} &  {mean, ms} &  {std, ms} &  {count}  \\
\midrule
kŭ.\textbf{ńaŋ} &       a &    \textbf{183.35} &    29.37 &      5  \\
ʔa.\textbf{ńaŋ} &       a &     72.00 &    17.67 &      3  \\
tă.n°.\textbf{šaŋ} &       a &     81.72 &    11.28 &      3  \\
\bottomrule
\end{tabular}
\end{table}
	
\end{frame}

\begin{frame}[fragile]{Качество гласных}

	\begin{enumerate}[\ding{72}]
		\item Иерархия сонорности: гласные выстраиваются в иерархию по качеству, где более высокая позиция связана с повышенной проминентностью, втч. с увеличенной длительностью \parencite{kenstowicz1997, delacy2002, parker2002}
		\item Ожидается, что длительность будет расти в порядке\\ high < mid < low
	\end{enumerate}
	
	\ex\label{ex:son-hierarchy}
		\textbf{Универсальная иерархия сонорности} \\ (\cite[162]{kenstowicz1997}, \cite[55]{delacy2002})\\
		\vspace*{.5em}
\begin{table}[H]
\begin{tabular}{ll}
low peripheral  & \emph{a}    \\[.1em]
mid peripheral  & \emph{e, o} \\[.1em]
high peripheral & \emph{i, u} \\[.1em]
mid central     & \emph{ə}    \\[.1em]
high central    & \emph{ɨ}   
\end{tabular}
\end{table}	
	\xe	
	
\end{frame}

\begin{frame}{Качество гласных под ударением}

	\begin{enumerate}[\ding{167}]
		\item Ожидается, что длительность будет расти в порядке\\ high < mid < low
		\item Под ударением, резкого эффекта иерархии нет
	\end{enumerate}
	
	\begin{figure}[H]
		\includegraphics[height=.55\paperheight]{stressed\_vowel\_quality\_dist}
	\end{figure}

\end{frame}

\begin{frame}{Качество безударных гласных}

	\begin{enumerate}[\ding{45}]
		\item В безударных слогах, разница также минимальна
	\end{enumerate}
	
	\begin{figure}[H]
		\includegraphics[height=.65\paperheight]{unstressed\_vowel\_quality\_dist}
	\end{figure}

\end{frame}


\begin{frame}{Длительности гласных: обобщения}

	\begin{enumerate}[\ding{166}]
		\item Долгие гласные более длинные
		\item Краткие гласные более короткие
		\item В безударных слогах, на длительности влияют два фактора:
			\vspace*{-1em}
			\begin{enumerate}[\ding{167}]
				\item В открытых слогах гласные длиннее, чем в закрытых (применимо и к ударным слогам)
				\item Конечные слоги в двусложных словах с начальным CV-слогом содержат длинные гласные
			\end{enumerate}
		\item Сонорность не имеет выраженного эффекта на длительности
	\end{enumerate}

\end{frame}

	\section{Обсуждение}

\begin{frame}{Длительность гласных и просодия}

	Акустические данные о длительностях гласных заставляют задуматься о понятии ударения в ЛН
	\vspace*{1em}
	
	\begin{enumerate}[\ding{36}]
		\item Ударение, согласно описаниям \parencite{sammallahti1974, salminen2007}, падает на нечетные нефинальные слоги
		\item В стандартном понимании, ударение увеличивает длительность, интенсивность и/или повышает тон гласной, также может сохранять сегментные контрасты
		\item В ЛН, ударная гласная может звучать короче безударной, если она краткая
		\item Гласная в безударном слоге может быть аналогична ударной долгой по длительности (в словах формы CV.CVVC)
	\end{enumerate}
	
	\vspace*{1em}
	Что значит быть ударной гласной?

\end{frame}

\begin{frame}{Что значит быть ударным?}

	Предположим стандартное определение ударения для ЛН
	\vspace*{1em}
	
	\begin{enumerate}[\ding{164}]
		\item Пусть ударение прямо коррелирует с длительностью: более длинные гласные ударные, а более короткие --- безударные
		\item Тогда мы сталкиваемся со ``третьей'' длительностью --- некоторые гласные по необъяснимой причине будут между ударными и безударными по длительности
	\end{enumerate}
		
	\vspace*{1em}
\begin{table}[H]
\centering
\begin{tabular}{llrrrl}
\toprule
word & segment &  mean, ms &  std, ms &  count & position \\
\midrule
\textbf{ka}.ta &       a &    139.14 &    32.91 &     32 &  initial \\
ka.\textbf{ta} &       a &     96.97 &    35.30 &     32 &    final \\[.5em]

\textbf{kă}.ta &       ă &     33.32 &    15.69 &     19 &  initial \\
kă.\textbf{ta} &       a &    110.81 &    43.05 &     17 &    final \\[.5em]

ka.\textbf{ta}.maʔ &       a &    108.46 &    18.46 &      6 &   medial \\
\bottomrule
\end{tabular}
\end{table}

\end{frame}

\begin{frame}{Что значит быть ударным?}

	Предположим стандартное определение ударения для ЛН
	\vspace*{1em}
	
	\begin{enumerate}[\ding{71}]
		\item Также мы будем часто наблюдать 2 ударных слога подряд, там, где стоит гласная шва
		\item Шва способна воспроизводить эффекты ударения даже на безударных слогах
		\item Ничто не запрещает иметь слово, состоящее исключительно из ударных слогов
	\end{enumerate}
	
	\vspace*{1em}
\begin{table}[H]
\centering
\begin{tabular}{llrrrl}
\toprule
  word & segment &  mean, ms &  std, ms &  count & position \\
\midrule
\textbf{ka}.ta.λ° &       a &    117.37 &    26.90 &     22 &  initial \\
ka.\textbf{ta}.λ° &       a &    141.09 &    34.93 &     21 &   medial \\
\bottomrule
\end{tabular}
\end{table}

\end{frame}

\begin{frame}{Что значит быть ударным?}

	Предположим стандартное определение ударения для ЛН
	\vspace*{1em}
	
	\begin{enumerate}[\ding{232}]
		\item Также мы будем часто наблюдать 2 ударных слога подряд, там, где стоит гласная шва
		\item Шва способна воспроизводить эффекты ударения даже на безударных слогах
		\item Ничто не запрещает иметь слово, состоящее исключительно из ударных слогов
	\end{enumerate}

\ex \colorbox{pink}{λa.t°}.\colorbox{pink}{ŋa.m°} \hfill `ударить.{\Gfs}.{\Obj}.{\Sg}.{\Fsg}'\\
	{[λatŋam]}\\[.5em]
	\colorbox{pink}{xæ.m°}.\colorbox{pink}{ceḿ}.\colorbox{pink}{ṕo.š°}.tu.mă.š° \hfill `видеть.{\Prog}.{\Hab}.{\Obj}.{\Sg}.{\Fsg}.{\Prt}'\\
	{[xæmčeḿṕoštumăš]}
\xe

\end{frame}

\begin{frame}{Что значит быть ударным?}

	С другой стороны, у картины мира, в которой в ЛН ударение падает на нечетные нефинальные слоги, есть свои проблемы
	\vspace*{1em}
	
	\begin{enumerate}[\ding{104}]
		\item Есть единицы, подверженные синкопированию кратких гласных в первом слоге
		
\ex
	pĭʰta $\sim$ pta \hfill `он/она/оно'\\
	čŭkˊi $\sim$ čkˊi \hfill `этот/эта/это'
\xe

		\item Найти выражение ударения в фонологии --- сложно \parencite{belov-shikunova2023, shikunova2024}, а его коррелят в фонетике --- невозможно
	\end{enumerate}

\end{frame}

%\begin{frame}{Благодарности}
%
%	\begin{enumerate}[$\gg$]
%		\item Всем консультантам, работавшим с нами в г. Тарко-Сале и с. Харампур
%		\item 
%	\end{enumerate}
%
%\end{frame}

\begin{frame}[allowframebreaks]
\frametitle{Источники}
%\bibliography{ref}
\printbibliography
\end{frame}

\section{Приложение}

\begin{frame}
\frametitle{Глоссы}
%\begin{multicols}{2}
\printglossary[title={}, style=mcolindex, nonumberlist]
%\end{multicols}
\end{frame}

\begin{frame}{Глоссарий примеров}

	    	\emph{kŭńaŋ} \hfill `куда' \\
	    	\emph{ʔańaŋ} \hfill `Аня.{\Gen}' \\
	    	\emph{tăn°šaŋ} \hfill `лестница' \\
    	\emph{kata} \hfill `бабушка'\\
	    	\emph{kăta} \hfill `ноготь'\\
    	\emph{kataλ°} \hfill `бабушка.{\Poss}.{\Ssg}'\\

\end{frame} 

\begin{frame}{Краткая гласная}
		
	\begin{figure}[H]
		\centering
		\includegraphics[width=\textwidth]{AOK_kăta}
		\label{}
		\caption*{\emph{kăta} `fingernail', tsOKT}
	\end{figure} 

\end{frame}

\begin{frame}{Долгая гласная}
		
	\begin{figure}[H]
		\centering
		\includegraphics[width=\textwidth]{AOK_kata_2}
		\label{}
		\caption*{\emph{kata} `grandmother', tsOKT}
	\end{figure}

\end{frame}

\end{document}